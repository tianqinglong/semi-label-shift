\documentclass[12pt]{article}
\usepackage{latexsym}
\usepackage{amsthm}
\usepackage{amsmath,amssymb,amsfonts,bm,amsbsy,bbm}
\usepackage{epsfig}
\usepackage{graphicx,rotating,lscape}
\usepackage{verbatim}
\usepackage{natbib}
\usepackage{multirow,color}
\usepackage{geometry}
\usepackage{setspace}
\usepackage{subfigure}
\usepackage{float}
\usepackage{physics}

\geometry{left=0.6in,right=0.6in,top=0.6in,bottom=0.6in}

\newtheorem{lemma}{Lemma}
\newtheorem{theorem}{Theorem}
\newtheorem{corollary}{Corollary}
\newtheorem{proposition}{Proposition}
\newtheorem{example}{Example}
\newtheorem{condition}{Condition}
\newtheorem{remark}{Remark}
\newtheorem{assumption}{Assumption}

\DeclareMathOperator{\sign}{sgn}

\def\A{{\bf A}}
\def\a{{\bf a}}
\def\B{{\bf B}}
\def\b{{\bf b}}
\def\C{{\bf C}}
\def\c{{\bf c}}
\def\D{{\bf D}}
\def\d{{\bf d}}
\def\e{{\bf e}}
\def\G{{\bf G}}
\def\g{{\bf g}}
\def\H{{\bf H}}
\def\h{{\bf h}}
\def\I{{\bf I}}
\def\M{{\bf M}}
\def\m{{\bf m}}
\def\Q{{\bf Q}}
\def\R{{\bf R}}
\def\S{{\bf S}}
\def\s{{\bf s}}
\def\T{{\bf T}}
\def\t{{\bf t}}
\def\U{{\bf U}}
\def\u{{\bf u}}
\def\V{{\bf V}}
\def\v{{\bf v}}
\def\W{{\bf W}}
\def\w{{\bf w}}
\def\X{{\bf X}}
\def\x{{\bf x}}
\def\Y{{\bf Y}}
\def\y{{\bf y}}
\def\Z{{\bf Z}}
\def\z{{\bf z}}

\def\calS{{\cal S}}
\def\calT{{\cal T}}
\def\calM{{\cal M}}

\def\ba{{\boldsymbol\alpha}}
\def\bb{{\boldsymbol\beta}}
\def\bg{\boldsymbol\gamma}
\def\bd{{\boldsymbol\delta}}
\def\bt{{\boldsymbol\theta}}
\def\bphi{\boldsymbol\phi}
\def\bmu{\boldsymbol\mu}
\def\btau{\boldsymbol\tau}
\def\bnu{\boldsymbol\nu}
\def\beps{\boldsymbol\epsilon}
\def\bta{\boldsymbol\eta}
\def\bl{\boldsymbol\lambda}
\def\bk{\boldsymbol\kappa}
\def\bpsi{\boldsymbol\psi}
\def\bxi{\boldsymbol\xi}
\def\bOmega{{\boldsymbol\Omega}}

\def\0{{\bf 0}}
\def\trans{^{\rm T}}
\def\pr{\hbox{pr}}
\def\wh{\widehat}
\def\wt{\widetilde}
\def\wc{\widecheck}
\def\N{{\rm N}}
\def\th{^{th}}
\def\var{\hbox{var}}
\def\cov{\hbox{cov}}
\def\cor{\hbox{cor}}
\def\eff{_{\rm eff}}
\def\opt{_{\rm opt}}
\def\sub{{\rm sub}}
\def\cat{{\rm cat}}
\def\alt{_{\rm alt}}
\def\n{\nonumber}
\def\dim{\mbox{dim}}
\def\MSE{\mbox{MSE}}
\def\rank{\mbox{rank}}
\def\vec{\mbox{vec}}
\def\argmin{\mbox{argmin}}
\def\argmax{\mbox{argmax}}
\def\diag{\mbox{diag}}
\def\tr{\mbox{trace}}
\def\dist{\hbox{dist}}
\def\dim{\hbox{dim}}
\def\Normal{\hbox{Normal}}
\def\ccdot{{\bullet}}
\def\log{{\rm log}}
\def\bias{\hbox{bias}}

\def\bse{\begin{eqnarray*}}
	\def\ese{\end{eqnarray*}}
\def\be{\begin{eqnarray}}
	\def\ee{\end{eqnarray}}
\def\bsq{\begin{equation*}}
	\def\esq{\end{equation*}}
\def\bq{\begin{equation}}
	\def\eq{\end{equation}}

\newcommand{\Eone}{E_1}
\newcommand{\Ezero}{E_0}

\def\fry{f_{R\mid Y}}
\def\fxyr{f_{\X\mid Y,R}}
\def\fyxrone{f_{Y\mid \X,1}}
\def\fyr{f_{Y\mid R}}
\def\fr{f_{R}}
\def\fy{f_{Y}}

\def\fyx{f_{Y\mid \X}}
\def\fxy{f_{\X\mid Y}}
\def\fx{f_{\X}}
\def\fxrone{f_{\X\mid 1}}
\def\fone{f_1}
\def\fzero{f_0}

\def\fryx{f_{R\mid Y, \X}}
\def\fryu{f_{R\mid Y, \U}}
\def\fyuz{f_{Y\mid \U,\Z}}
\def\fuz{f_{\U\mid \Z}}
\def\fz{f_{\Z}}
\def\fzu{f_{\Z\mid \U}}
\def\fzur{f_{\Z\mid \U,R_\U=1}}
\def\fu{f_{\U}}
\def\fyu{f_{Y\mid \U}}

\def\sumi{\sum_{i=1}^n}
\def\sumj{\sum_{j=1}^n}
\def\sumI{\sum_{i=1}^N}
\def\sumIP1{\sum_{i=1, i\in P_1}^N}
\def\sumJ{\sum_{j=1}^N}
\def\sumK{\sum_{k=1}^N}
\def\suml{\sum_{l=1}^n}
\def\sumk{\sum_{k=1}^n}
\def\prodI{\prod_{i=1}^N}
\def\prodi{\prod_{i=1}^n}

\def\boxit#1{\vbox{\hrule\hbox{\vrule\kern6pt\vbox{\kern6pt#1\kern6pt}\kern6pt\vrule}\hrule}}
\def\jiwei#1{\vskip 2mm\boxit{\vskip 2mm{\color{blue}\bf#1} {\color{blue}\bf -- Jiwei\vskip 2mm}}\vskip 2mm}
\def\qt#1{\vskip 2mm\boxit{\vskip 2mm{\color{red}\bf#1} {\color{red}\bf -- Qinglong\vskip 2mm}}\vskip 2mm}

\newcommand{\jiweizhao}[1]{{\color{blue}{[Jiwei to himself: #1]}}\\}

\usepackage{tikz}
\usetikzlibrary{positioning,shapes.geometric}

\makeatletter
\newcommand*{\ind}{%
	\mathbin{%
		\mathpalette{\@ind}{}%
	}%
}
\newcommand*{\nind}{%
	\mathbin{% % The final symbol is a binary math operator
		\mathpalette{\@ind}{\not}% \mathpalette helps for the adaptation
		% of the symbol to the different math styles.
	}%
}
\newcommand*{\@ind}[2]{%
	% #1: math style
	% #2: empty or \not
	\sbox0{$#1\perp\m@th$}% box 0 contains \perp symbol
	\sbox2{$#1=$}% box 2 for the height of =
	\sbox4{$#1\vcenter{}$}% box 4 for the height of the math axis
	\rlap{\copy0}% first \perp
	\dimen@=\dimexpr\ht2-\ht4-.2pt\relax
	% The equals symbol is centered around the math axis.
	% The following equations are used to calculate the
	% right shift of the second \perp:
	% [1] ht(equals) - ht(math_axis) = line_width + 0.5 gap
	% [2] right_shift(second_perp) = line_width + gap
	% The line width is approximated by the default line width of 0.4pt
	\kern\dimen@
	{#2}%
	% {\not} in case of \nindep;
	% the braces convert the relational symbol \not to an ordinary
	% math object without additional horizontal spacing.
	\kern\dimen@
	\copy0 % second \perp
}
\makeatother

\title{Simulation Report}
\author{Qinglong Tian}
\date{updated on \today}

\begin{document}

\maketitle

\section*{Setting~1}

\subsection*{Data Generating Mechanism}

The joint distribution of $(Y,\X)\in\mathbb{R}^4$ from the \textbf{source} population has a multivariate normal distribution, which is given by
\[
(Y,\X)\trans\sim\mathrm{MVN}(\mu_{Y\X}, \Sigma_{Y\X}),
\]
where
\[
\mu_{Y\X}=(2,1,1,1)\trans,\quad\Sigma_{Y\X}=\begin{bmatrix} 1.44 & 0.9 & 0.81 & 0.729 \\\\ 0.9 & 1.1 & 0.3 & 0.09 \\\\ 0.81 & 0.3 & 1.1 & 0.3 \\\\ 0.729 & 0.09 & 0.3 & 1.1 \end{bmatrix}.
\]
The condition distribution of $Y$ given $\X$ on the source population is given by
\[
p_s(y|\x)\sim\mathrm{Norm}(\alpha_0+\x\trans\ba_1, \sigma^2),
\]
where
\[
\alpha_0=0.422,\quad\ba_1=(0.663, 0.421, 0.494)\trans,\quad\sigma^2=0.142.
\]
The conditional distribution of $\X$ given $Y$ is given by
\[
p(\x|y)\sim\mathrm{MVN}(\mu_{\X|Y}, \Sigma_{\X|Y}),
\]
where
\[
\mu_{\X|Y}=\begin{bmatrix}
	-0.25\\-0.125\\-0.0125
\end{bmatrix}+
\begin{bmatrix}
	0.625\\0.5625\\0.50625
\end{bmatrix}y,\quad\Sigma_{\X|Y}=\begin{bmatrix} 0.5375 & -0.20625 & -0.365625 \\\\ -0.20625 & 0.644375 & -0.1100625 \\\\ -0.365625 & -0.1100625 & 0.73094375 \end{bmatrix}.
\]

The marginal distribution of $Y$ on the target distribution is
\[
p_t(y)\sim\mathrm{Norm}(\mu=1.5, \sigma^2=2.25).
\]
Consequently, function $\rho(y;\bb)$ is given by
\[
\rho(y;\bb)=\beta_1y+\beta_2y^2,
\]
where
\[
\beta_1=-0.722,\quad\beta_2=0.125.
\]

The sample size of the source distribution data is denoted by $n$ while the sample size of the target distribution data is denoted by $m$.
The proportion $\pi$ is computed by $\pi=n/(n+m)$.
The Monte Carlo sample size is $B=2000$.
In this simulation, we estimate $\bb$ using both the true $p_s(y|\x)$ and the fitted (but correctly specified) $\widehat p_s(y|\x)$.

\subsection*{Simulation Results}

The simulation results using the true model $p_s(y|\x)$ are given in Table~\ref{Table:Set1-True}.
\begin{table}[ht]
	\centering
		\begin{tabular}{rrrrrrrrrr}
			\hline\hline
			\multicolumn{1}{c|}{\multirow{2}{*}{n}} & \multicolumn{1}{c|}{\multirow{2}{*}{m}} & \multicolumn{4}{c|}{$\beta_1$}                                                                      & \multicolumn{4}{c}{$\beta_2$}                                                                      \\ \cline{3-10} 
			\multicolumn{1}{c|}{}                  & \multicolumn{1}{c|}{}                  & \multicolumn{1}{c}{Mean} & \multicolumn{1}{c}{Bias} & \multicolumn{1}{c}{SE} & \multicolumn{1}{c|}{SD} & \multicolumn{1}{c}{Mean} & \multicolumn{1}{c}{Bias} & \multicolumn{1}{c}{SE} & \multicolumn{1}{c}{SD} \\ \hline\hline
			500 &    500                                    &         -0.680             &           0.0427           &          0.113            &             3.425          &         0.119             &            -0.00630          &           0.0286           &         1.355             \\
			500 &     1000                                   &      -0.717               &       0.00556               &      0.123                &         3.679              &       0.116               &       -0.00857              &        0.04436              &          1.529           \\
			1000&       500                                 &       -0.703              &            0.0189          &          0.114           &          3.444          &       0.121             &       -0.00392             &            0.0266         &        1.351            \\
			1000&         1000                               &         -0.687             &        0.0352              &       0.0943               &          3.410            &           0.119           &        -0.00617              &            0.0198          &          1.340            \\
			1500&         1500                               &       -0.691               &          0.0315            &   0.0824                   &       3.412                &       0.120               &      -0.00544                &           0.0169           &         1.342             \\ \hline\hline
		\end{tabular}
	\caption{Estimation of $\bb$ using the true model $p_s(y|\x)$.}
	\label{Table:Set1-True}
\end{table}

The simulation results using the fitted mode $\hat{p}_s(y|\x)$ are given in Table~\ref{Table:Set1-Fitted}.
\begin{table}[ht]
	\centering
	\begin{tabular}{rrrrrrrrrr}
		\hline\hline
		\multicolumn{1}{c|}{\multirow{2}{*}{n}} & \multicolumn{1}{c|}{\multirow{2}{*}{m}} & \multicolumn{4}{c|}{$\beta_1$}                                                                      & \multicolumn{4}{c}{$\beta_2$}                                                                      \\ \cline{3-10} 
		\multicolumn{1}{c|}{}                  & \multicolumn{1}{c|}{}                  & \multicolumn{1}{c}{Mean} & \multicolumn{1}{c}{Bias} & \multicolumn{1}{c}{SE} & \multicolumn{1}{c|}{SD} & \multicolumn{1}{c}{Mean} & \multicolumn{1}{c}{Bias} & \multicolumn{1}{c}{SE} & \multicolumn{1}{c}{SD} \\ \hline\hline
		500 &    500                                    &         -0.681             &           0.04173           &          0.112            &            3.420          &         0.119             &            -0.00633          &           0.0284           &         1.356             \\
		500 &               1000                         &       -0.717               &             0.00538        &      0.122              &           3.684         &         0.117            &       -0.00846            &          0.0445          &       1.531             \\
		1000 & 500                                      &        -0.702              &            0.0204          &          0.112           &            3.445          &         0.121             &        -0.00425              &            0.0262          &          1.352                    \\
		1000&                1000                        &         -0.689             &         0.0333             &         0.0954             &       3.414                &        0.119              &        -0.00609              &         0.0198             &          1.342            \\
		1500 &             1500                           &       -0.690              &      0.0322                &   0.0820                  &         3.414             &      0.119                &       -0.00556               &        0.0168             &         1.343             \\ \hline\hline
	\end{tabular}
	\caption{Estimation of $\bb$ using the fitted model $\widehat p_s(y|\x)$.}
	\label{Table:Set1-Fitted}
\end{table}

\subsection*{Discussions}
This simulation study shows that the estimation approach gives good results as the bias is generally small and the standard error is reasonable.
In both tables, SE is the standard error and SD is the mean of all $B=2000$ estimated standard deviations.
It is obvious that either my derivation/programming is incorrect or the formula does not work.
I checked the functions to compute $\mathrm{E}(\S\eff\S\eff\trans)$ and did not find anything.
It is of interest to investigate the effect of the ratio of $m/n$ on the Bias/SE in the following simulation.

I used $\mathrm{optim}()$ function for simplicity.
The running time is not fast but not too terrible as each row takes at most a few hours to run on the laptop.
If we want to run large number of $n$ and $m$, I can write the optimization algorithm instead of using $\mathrm{optim}()$.

\section*{Setting~2: Binary Response}

%Letting the support of $Y$ be either 0 or 1.
%Then the function $\rho(y;\bb)$ reduces to
%\[
%\rho(y;\bb)=\begin{cases}
%	\beta_0,\quad y=0.\\
%	\beta_1,\quad y=1.
%\end{cases}
%\]
%Define $\mathrm{C}\equiv\beta_0p_s(0)+\beta_1p_s(1)$, then
%\[
%\frac{p_t(i)}{p_s(i)}=\frac{\beta_i}{\mathrm{C}},\quad i=0,1.
%\]
%
%\qt{What we are interested is $\beta_i/\mathrm{C}$ rather than $\beta_i$.
%When making inference about $\beta_i$, can we directly scale the results by $1/\widehat{\mathrm{C}}$?
%If not, what we really interested in is the asymptotic variance of $\widehat{\beta}_i/\widehat{\mathrm{C}}$.
%This question is also applicable to the continuous case.}

\subsection*{Preparations}

We first generate a sample of $(Y,\X)$ from a distribution.
Then for each observation in this sample, we use a logistic regression model to determine if it belongs to the source or the target distribution.
We have a joint distribution for $(Y,\X,R)$, and we have
\[
\begin{split}
p\left(Y|\X,R=1\right)=&\frac{p(Y,\X,R=1)}{p(\X,R=1)}\\
=&\frac{p(R=1|Y,\X)p(Y|\X)p(\X)}{p(\X,Y=1,R=1)+p(\X,Y=0,R=1)}\\
=&\frac{p(R=1|Y,\X)p(Y|\X)p(\X)}{p(R=1|Y=1)p(Y=1|\X)p(\X)+p(R=1|Y=0)p(Y=0|\X)p(\X)}\\
=&\frac{\Pr(R=1|Y)\Pr(Y|\X)}{\Pr(R=1|Y=1)\Pr(Y=1|\X)+\Pr(R=1|Y=0)\Pr(Y=0|\X)}.
\end{split}
\]
Letting the ``mixed'' conditional distribution of $Y$ given $\X$ be
\[
\Pr(Y=1|\X=\x)=\frac{1}{1+\exp(-\ba\trans\x)},
\]
and
\[
\Pr(R=1|Y=y)=\frac{1}{1+\exp(-\gamma_0-\gamma_1y)}.
\]
Then $\Pr(Y|\X,R=1)$ reduces to
\[
\begin{split}
\Pr(Y=1|\X,R=1)&=\frac{\dfrac{1}{1+\exp(-\gamma_0-\gamma_1)}\dfrac{1}{1+\exp(-\ba\trans\x)}}{\dfrac{1}{1+\exp(-\gamma_0-\gamma_1)}\dfrac{1}{1+\exp(-\ba\trans\x)}+\dfrac{1}{1+\exp(-\gamma_0)}\dfrac{\exp(-\ba\trans\x)}{1+\exp(-\ba\trans\x)}}\\
&=\frac{1}{1+\dfrac{1+\exp(-\gamma_0-\gamma_1)}{1+\exp(-\gamma_0)}\exp(-\ba\trans\x)}.
\end{split}
\]
If we denote
\[
\alpha_0\equiv-\log\left\{\dfrac{1+\exp(-\gamma_0-\gamma_1)}{1+\exp(-\gamma_0)}\right\},
\]
then we have
\[
\Pr(Y=1|\X,R=1)=\frac{1}{1+\exp(-\alpha_0-\ba\trans\x)},
\]
which indicates that to correctly specify the model $p_s(y|\x)$, we can still use the logistic regression.

We are also interested in
\[
\begin{split}
\frac{p_t(y)}{p_s(y)}=&\frac{\Pr(y|R=0)}{\Pr(y|R=1)}=\frac{\Pr(y,R=0)/\Pr(R=0)}{\Pr(y,R=1)/\Pr(R=1)}=\frac{\Pr(R=1)}{\Pr(R=0)}\frac{\Pr(R=0|y)}{\Pr(R=1|y)}\\
=&\frac{\Pr(R=1)}{\Pr(R=0)}\exp(-\gamma_0)\exp(-\gamma_1y)\equiv\mathrm{C}_0\exp(-\gamma_1y).
\end{split}
\]
Letting $\rho(y;\beta)=\exp(\beta y)$, then we define $\mathrm{C}_1\equiv1/\left\{{\Pr}_{s}(Y=0)+{\Pr}_{s}(Y=1)\exp(\beta)\right\}$.
Thus, we have
\[
\mathrm{C}_0\exp(-\gamma_1 y)-\mathrm{C}_1\exp(\beta y)=0,
\]
for $y=0,1$.
This result indicates that $\beta=-\gamma_1$.

\subsection*{Data Generating Mechanism}



\newpage
\bibliographystyle{agsm}
\bibliography{ref}
\end{document}
